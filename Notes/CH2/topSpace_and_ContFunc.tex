\documentclass{article}
\usepackage{geometry}
\usepackage{xcolor}
\usepackage{graphicx}
\usepackage{mathtools}
\usepackage{enumitem}
\usepackage{amsmath,amsthm}
\usepackage[english]{babel}
\usepackage{amsfonts}
\usepackage{thmtools}
\usepackage{fancyhdr}
\declaretheorem{theorem}
\usepackage{rsfso}
\declaretheorem[sibling=theorem]{Definition}
\graphicspath{{./IMG/}}

\newcommand{\Lim}[1]{\raisebox{0.5ex}{\scalebox{0.8}{$\displaystyle \lim_{#1}\;$}}}
\usepackage{color} 
\usepackage{hyperref}
\hypersetup{
    colorlinks=true, %set true if you want colored links
    linktoc=all,     %set to all if you want both sections and subsections linked
    linkcolor=blue,  %choose some color if you want links to stand out
}

%-------------- HEADER -- FOOTER --------------------------------------------------------
% Assignment // project etc
\newcommand{\hmwkTitle}{Notes on Munkres Topology}
\newcommand{\hmwkAuthorName}{Noah Pinel}

%
% Title Page
%
\title{
    \vspace{2in}
    \textmd{\textbf{\hmwkTitle}}\\
    \author{Created by \hmwkAuthorName}
    \date{}
    \vspace{3in}
}

%--- LAZY -----------------------------------------------------------------------------------
\newcommand{\R}{\mathbb{R}}
\newcommand{\ps}{\{p_n\}}

%---------------------HEADER FOOTER-----------------------------------------------------------
\pagestyle{fancy}
\fancyhead{}
\fancyhead[L]{\slshape\nouppercase{\leftmark}}
\fancyfoot{} 
\newcommand\invisiblesection[1]{%
  \refstepcounter{section}%
  \addcontentsline{toc}{section}{\protect\numberline{\thesection}#1}%
  \sectionmark{#1}}
%--------------------------------------------------------------------------
\begin{document}
\maketitle % Print the title
%------ Hide pg # from title page -----------------
\thispagestyle{empty}
\newpage


\fancyfoot[R]{\thepage}
%----------------TABLE OF CONTENTS------------------------------------------
\tableofcontents
\invisiblesection{Table of contents}
\newpage
%----------------ABSTRACT---------------------------------------------------
\section{Abstract}
\begin{abstract}
    In a valiant effort to stop my Topology knowledge from dissipating,
    I have decided to self-study Munkres until my next Analysis course, which will be in about a years time.
    I know that if I don't keep my math skills sharp, they'll end up as flat as a non-differentiable function. 
    Hopefully this self-study will help me retain my math skills like a homeomorphism 
    preserves topological properties ;)



\end{abstract}
\newpage
%----------------CH 2---------------------------------------------------------
\section{Topological Spaces}
\newpage
\section{Basis for a Topology}
\newpage

\section{The Order Topology}
\newpage

\section{The Product Topology on \texorpdfstring{$X\times Y$}{TEXT}}
\newpage

\section{The Subspace Topology}
\newpage

\section{Closed Sets and Limit Points}
\newpage

\section{Continuous Functions}
\newpage

\section{The Product Topology}
\newpage

\section{The Metric Topology}
\newpage

\section{The Quotient Topology}
\newpage

\section{Topological Groups} 



\end{document}
