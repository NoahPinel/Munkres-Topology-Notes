\documentclass{article}
\usepackage{geometry}
\usepackage{xcolor}
\usepackage{graphicx}
\usepackage{mathtools}
\usepackage{enumitem}
\usepackage{amsmath,amsthm}
\usepackage[english]{babel}
\usepackage{amsfonts}
\usepackage{thmtools}
\usepackage{fancyhdr}
\declaretheorem{theorem}
\usepackage{rsfso}
\declaretheorem[sibling=theorem]{Def}
\graphicspath{{./IMG/}}
\usepackage{tcolorbox}
\tcbuselibrary{theorems}

% ----------------------- THEO, LEMMA, DEF -----------------------
\newtcbtheorem[number within=section]{myDef}{Def}%
{colback=blue!5,colframe=blue!35!black,fonttitle=\bfseries}{df}

\newtcbtheorem[number within=section]{myTheo}{Theorem}%
{colback=cyan!5,colframe=cyan!35!black,fonttitle=\bfseries}{th}

\newtcbtheorem[number within=section]{myLem}{Lemma}%
{colback=green!5,colframe=green!35!black,fonttitle=\bfseries}{lm}

\newtcbtheorem[number within=section]{myExa}{Example}%
{colback=black!5,colframe=gray!35!black,fonttitle=\bfseries}{ex}


% ----------------------- Index handle -----------------------
\newcommand{\Lim}[1]{\raisebox{0.5ex}{\scalebox{0.8}{$\displaystyle \lim_{#1}\;$}}}
\usepackage{color} 
\usepackage{hyperref}
\hypersetup{
    colorlinks=true, %set true if you want colored links
    linktoc=all,     %set to all if you want both sections and subsections linked
    linkcolor=blue,  %choose some color if you want links to stand out
}

%-------------- HEADER -- FOOTER --------------------------------------------------------
% Assignment // project etc
\newcommand{\hmwkTitle}{Munkres Topology Notes\\ \large Topological Spaces and Continuous Functions}
\newcommand{\hmwkAuthorName}{Noah Pinel}

%
% Title Page
%

\title{
    \vspace{2in}
    \textmd{\textbf{\hmwkTitle}}\\
    \author{Created by \hmwkAuthorName}
    \date{}
    \vspace{3in}
}

%--- LAZY -----------------------------------------------------------------------------------
\newcommand{\R}{\mathbb{R}}
\newcommand{\ps}{\{p_n\}}
\newcommand{\topo}{\mathcal T}

%---------------------HEADER FOOTER-----------------------------------------------------------
\pagestyle{fancy}
\fancyhead{}
\fancyhead[L]{\slshape\nouppercase{\leftmark}}
\fancyfoot{} 
\newcommand\invisiblesection[1]{%
  \refstepcounter{section}%
  \addcontentsline{toc}{section}{\protect\numberline{\thesection}#1}%
  \sectionmark{#1}}
%--------------------------------------------------------------------------
\begin{document}
\maketitle % Print the title
%------ Hide pg # from title page -----------------
\thispagestyle{empty}
\newpage


\fancyfoot[R]{\thepage}
%----------------TABLE OF CONTENTS------------------------------------------
\tableofcontents
\invisiblesection{Table of contents}
\newpage
%----------------ABSTRACT---------------------------------------------------
\section{Abstract}
\begin{abstract}
    In a valiant effort to stop my Topology knowledge from dissipating,
    I have decided to self-study Munkres until my next Analysis course, which will be in about a years time.
    I know that if I don't keep my math skills sharp, they'll end up as flat as a non-differentiable function. 
    Hopefully this self-study will help me retain my math skills like a homomorphism 
    preserves topological properties ;)



\end{abstract}
\newpage
%----------------CH 2.1---------------------------------------------------------
\invisiblesection{Topological Spaces}
% \vspace*{1cm}
\begin{center}
    \title{\textbf{\LARGE \underline{Topological Spaces}}}
\end{center}
% \vspace*{2cm}
\subsection{What is a Topology?}
\begin{myDef*}{(Topology)}
    A topology on a set X is a collection $\topo$ of subsets of X having 
    the following properties:
        \begin{enumerate}[label=\roman*.]
            \item $\emptyset$ and X are both in $\topo$
            \item The union of the elements of any subcollection of
                  $\topo$ is in $\topo$
            \item The intersection of the elements of any finite
                  collection of $\topo$ is in $\topo$
        \end{enumerate}
\end{myDef*}
\noindent
If a set X satisfies the above three properties we call X a topological 
space.
\subsection{Open Sets} 
\begin{myDef*}{(Open Set)}
    If X is a topological space with a topology $\topo$, we
	say that a subset U of X is an open set of X if U belongs to the 
	collection of $\topo$.
\end{myDef*}
The whole idea here is to have this notion of openness be very general,
the above definition is literally just saying that a set is open if 
it is a member of $\topo$. YOU choose what is open by placing sets in
$\topo$, the only catch is that what you decide to declare open must 
satisfy the requirements for a topology; that is, $\emptyset$ and X are 
in $\topo$ and $\topo$ is closed under arbitrary union and finite 
intersection. To tie together the notion of a topological space nicely 
Munkres has a very nice way of summarizing the concept of a topological
space using the new vocabulary we have learned.

\subsection{A more Concise Definition for Topological Space's} 
\begin{myDef*}{(Concise Definition for Topological Space's)}
    A topological space is a set X together with a collection of subsets
    of X, called open sets, such that $\emptyset$ and X are both open, and
    such that the arbitrary union and finite intersection of open sets are open.
\end{myDef*}

\begin{myExa*}{}
    \begin{itemize}
        \item Choose X to be any set, the collection of all subsets of X, i.e., the power set of X denoted 
              $\mathcal{P} (X)$ is a topology, it is called the discrete topology.\\
        \item The collection of only X and $\emptyset$ is a topology on X, it is called the indiscrete topology,
              or more commonly just the trivial topology.
    \end{itemize}
\end{myExa*}
\noindent
We will now describe some terms for when we compare different topologies.
\subsection{Fine and Course Topologies} 
\begin{myDef*}{(Fine and Course Topologies)}
    Let $\topo$ and $\topo^{'}$ be two topologies on a set X. If $\topo^{'} \supset \topo$, we say that $\topo^{'}$ is 
    finer than $\topo$. If $\topo^{'}$ properly contains $\topo$, we say that $\topo^{'}$ is strictly finer than $\topo$.
    One can also say that $\topo$ is courser than $\topo^{'}$, or strictly courser than $\topo^{'}$,in the two respective
    cases.
\end{myDef*}
\subsection*{Comparable Topologies}
\begin{myDef*}{(Comparable)}
    We say $\topo$ is comparable to $\topo^{'}$ if either $\topo^{'} \supset \topo$ or $\topo \supset \topo^{'}$. 
\end{myDef*}
\noindent
Note, We can also use the terms
\begin{itemize}
    \item ``Weak topology'' when $\topo^{'} \supset \topo$.
    \item ``Strong topology'' when $\topo \supset \topo^{'}$. 
\end{itemize}
\noindent
There tends to be some miscommunication as to the correct definitions of weak and strong topologies. I use the above
associations because that is how I learned them, some people define weak/strong in the opposite way, so for future reading
make sure you understand how one is defining there weak/strong topologies. Similarly, we can say that a topology is ``Larger''
or ``Weaker'' with the above definitions/caution being the same.
\noindent
To finish this portion of chapter 2 I'll provide a rather straightforward example that illustrates the above ideas.
\begin{myExa*}{}
    Let
    \begin{center}
        $X=\{a,b,c\},\quad \topo_1 =\{\emptyset,\ X\},\quad  \topo_2 =\mathcal{P} (X)$
    \end{center}
    \begin{itemize}
        \item It's simple to see that $\topo_1$ is just the trivial topology, we can say that $\topo_1$ is the ``Coarsest'' topology.
        \item We call $\topo_2$ the ``Finest'' topology.
        \item Lastly note that $\topo_1 \subset \topo_2$, thus $\topo_1$ is coarser than $\topo_2$ and
            $\topo_2$ is finer than $\topo_1$.
    \end{itemize}
    
\end{myExa*}
\newpage

%----------------CH 2.2---------------------------------------------------------
\invisiblesection{Basis for a Topology}
\begin{center}
    \title{\textbf{\LARGE \underline{Basis for a Topology}}}
\end{center}
\subsection{Bases}
\begin{myDef*}{(bases)}
    Let X be a set, a basis for a topology on X is a
	collection $\mathcal{B}$ of subsets of X,(Called 
	basis elements) such that,
	\begin{itemize}
		\item For each $x\in X$, there is at least one basis 
		      element $\mathcal{B}$ containing x.
		\item If x belongs to the intersection of two basis
		      elements $\mathcal{B}_1$ and $\mathcal{B}_2$, 
		      then there is a basis element $\mathcal{B}_3$
		      containing x such that $\mathcal{B}_3 \subset
		      \mathcal{B}_1 \cap \mathcal{B}_2$.
		      	 
	\end{itemize}
\end{myDef*}
\noindent If $\mathcal{B}$ satisfies the above two properties, then we 
define the topology $\topo$ generated by $\mathcal{B}$ as follows, A subset
U of X is said to be open in X (that is, to be an element of $\topo$) if
for each $x\in U$, there is a basis element $B\in \mathcal{B}$ such that
$x\in B$ and $B \subset U$. Note that each basis element is itself in
$\topo$.
\newpage

%----------------CH 2.3---------------------------------------------------------
\section{The Order Topology}
\newpage

\section{The Product Topology on \texorpdfstring{$X\times Y$}{TEXT}}
\newpage

\section{The Subspace Topology}
\newpage

\section{Closed Sets and Limit Points}
\newpage

\section{Continuous Functions}
\newpage

\section{The Product Topology}
\newpage

\section{The Metric Topology}
\newpage

\section{The Quotient Topology}
\newpage

\section{Topological Groups} 



\end{document}
